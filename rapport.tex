\documentclass[a4paper,oneside,12pt]{article}

\usepackage[utf8x]{inputenc}
\usepackage[T1]{fontenc}
\usepackage{hyperref}
\usepackage{eurosym}
\usepackage{graphicx}
\usepackage{array}
\newcolumntype{C}{>{\centering\arraybackslash}p{8ex}}
\usepackage[binary-units = true]{siunitx}

\title{TASBot64\\Specifications}
\author{rcavadas, abureau,\\cmutti, lgillot-}

\begin{document}
\maketitle

\section{Functional purpose}
The TASBot64 is a device that plugs in a Nintendo 64 to play a pre-recorded Tool
Assisted Speedrun (TAS) input file, faking a gamepad.

A TAS is a speedrun (the completion of a video game in the shortest time
possible) whose inputs, that is to say the sequence of buttons pressing and
movements on the gamepad, are given frame-by-frame on an emulator.

Emulators enable techniques such as frame-by-frame operation, precise digital
and analog inputs, re-recording (saving and reloading the entire state of the
game at any moment) or even memory watching and disassembly.

Thanks to this, TASs showcase feats unattainable to the human player and often
feature blatant abuse of the game mechanisms, exploiting bugs in the games that
was not intended to receive such fast and improbable input.

If the emulator is needed to craft the input file, the sequence of inputs itself
however is perfectly valid and theorically could come from a real gamepad. Hence
the goal of the TASBot64, that is to perform a TAS on a real Nintendo 64 instead
of an emulator, thus validating that the TAS does not depend on any emulator
specificity.

\subsection{Features}
\begin{itemize}
\item Behave as a genuine Nintendo 64 gamepad
\item Read and play the Mupen64-rr input file format from a SD card
\item Allow selecting between different files on the SD
\item Allow selecting between 50Hz (Europe) and 60Hz (US and Japan) operation
\item Display the name of the current selected file
\item Display errors or unsupported features if they occur
\item Allow passthrough of a gamepad to the console to avoid frequent unplugging
\item Take its power from the Nintendo 64 (no power source required)
\end{itemize}

\subsection{Performances}
\begin{itemize}
\item The user must be able to choose between 4 different input files
\item The TASBot64 must support input files as big as 4GB
\item The output stream to the console must be fast enough so as to not miss a
  game frame: a complete input frame must be emited in less than a 60th of a
  second.
\item The output stream rate must not drift more than half a frame (at most 8ms)
  in a runtime of 24 hours
\item The delay between startup of console and first output frame must be
  shorter than 0.5s and must not vary more than a quarter of a frame (at most
  4ms) between runs
\item The latency between input and output during passthrough must be less than
  1 ms
\end{itemize}

\section{User manual}
\subsection{Foreword}

\subsection{Package content}

\subsection{Installation}
The TASBot64 is made up by two not separable parts: the PCB and its wire on the
free-end of which is a male controller adaptor that enables the connection
between the TASBot64 and the Nintendo 64 system.

The PCB itself presents on its control panel:
\begin{itemize}
\item Two multiple-selector buttons:
\begin{itemize}
\item The "Mode" button which allows you to chose between "PASSTHROUGH" mode and
"TAS" mode
\item The "Game selector" button which lets you toggle between the (up to) four
different files you have loaded on your SD card
\end{itemize}

\item A LCD screen which displays:
\begin{itemize}
\item The name of the file you are currently playing (if in "TAS" mode)
\item The name of the files while changing the game selector position
\item Error messages (when occurs)
\end{itemize}

\item Two LEDs:
\begin{itemize}
\item A "Power" LED that lights on if the TASBot is correctly connected to the
Nintendo 64 system and if it is on.
\item A "Data" LED that flashes when data is sent to the console.
\end{itemize}
\end{itemize}

You can also find two ports:
\begin{itemize}
\item The SD port where you can insert your SD card (non included).
\item The controller port that enables you to plug a controller in case you want
to play the game instead of launching a TAS (remember to switch the "Mode"
button to the "PASSTHROUGH" mode in that case).
\end{itemize}

\subsection{Reading a TAS}
\subsubsection{Getting files (Mupen64 movies)}
First of all, you have to download at least one and up to four files to be read
by your TASBot64 from the Internet to your SD card (not included). Such files
can be found on specialized websites such as tasvideos.org
Once the files are loaded, insert your SD card into the adapted slot of the
TASBot64.

Our tip: you may want to visit http://tasvideos.org/Movies-N64-Stars-Moons.html

Important: please notice that you must own the cartridge of the game you want to
run the TAS.

Important: please notice that only .m64 files (Mupen64 movies) can be read by
your TASBot64 so make sure to get these.

\subsubsection{Preparing the ground}
Every TAS has been created to run from a certain state of the game. You have to
put your game cartridge in this specific state in order to run the TAS you've
chosen. Generally this means deleting all data from the first file of the game
cartridge but THIS MAY VARY FOR SOME OF THEM. In any case, we recommend to
double check the TAS video (you will find links to these under the .m64 files on
tasvideos.org) before starting a run.

\subsubsection{Running the TAS}
\begin{itemize}
\item Put the appropriate game cartridge into your Nintendo 64 system
\item Connect your TASBot64 to the first controller port of your Nintendo 64
system
\item Switch the mode switch to TAS position
\item Switch the game selector switch to the appropriate position, regarding to
the partition in which you uploaded the .m64 file to. The name of the selected
game will briefly appear on the LCD screen if the Nintendo 64 system is on.
\item Turn your Nintendo 64 system on. If it is already on, then turn it off and
on again and enjoy the watch
\end{itemize}

\subsection{Passthrough mode}

\subsection{Errors and how to fix them}
\subsubsection{TAS mode}
\begin{itemize}
\item TAS doesn't start: check if the mode switch is on TAS mode
\item TAS seems desynchronized: check if the game selector switch is at the
correct position, then restart the Nintendo 64 system
\item TAS didn't finish the game: the game cartridge may not be at its correct
initial state or the TAS may be invalid.
\end{itemize}

\subsubsection{Passthrough mode}
\begin{itemize}
\item The game seems to play by itself: check if the mode switch is on
Passthrough mode
\item I can't do anything : Check if the controller is properly connected
\end{itemize}

\subsubsection{Error Code}
\begin{itemize}
\item Error 000: insert a SD card in the adapted slot
\item Error 001: error while parsing the TAS file, make sure that the file
system of your SDCARD is fat32 and check your TAS file type (it must be .m64)
\item Error 002: SD card is empty, load at least one file in your SD card
\end{itemize}

\section{Protocols}
\subsection{N64 to Gamepad}
\subsubsection{Electrical levels}
The gamepad is connected to the Nintendo 64 by a cable made of a \SI{3.3}{\V}
alim wire, a ground wire and a data wire.

This data wire is a bidirectionnal serial bus. It is operated in an open
collector way, pulled up to \SI{3.3}{\V}. Hence, both the console and the
gamepad can talk half-duplex on it by pulling down the line in a wire-AND
scheme.

As measured, the console pulls down the line to \SI0{\V}, while our gamepad
pulls it only down to \SI{655}{\mV}, probably due to a diode on the path.

\subsubsection{Bit signaling and framing}
Each bit of data on the wire is made of a falling edge followed by a rising
edge. What distinguish a 0 bit from a 1 bit is the duration of the low state, as
pictured in figures \ref{0bit} and \ref{1bit}.

\begin{figure}
  \includegraphics[width=\textwidth]{0bit.png}
  \caption{0 bit}
  \label{0bit}
\end{figure}

\begin{figure}
  \includegraphics[width=\textwidth]{1bit.png}
  \caption{1 bit}
  \label{1bit}
\end{figure}

Communication is done in frames of 1 to 4 bytes. A frame starts directly with a
data bit. Each bit is separated of the next by a high state of at least
\SI{0.5}{\us}. After all bytes are transmitted, the frame ends with a last
falling edge and then the line is released.

\subsubsection{Input protocol}
The inputs are always transmitted on console demand: the gamepad never
spontaneously emit input information. When the console sends a request frame,
the gamepad sends back a frame containing the full state of the gamepad at the
moment. The total communication does not take longer than \SI{200}{\us}.

The console does not necessarily asks for input on a regular basis: a long time
could go between requests, for example if the game currently does not take
inputs at the moment or if it taking inputs from another gamepad. However the
console will not ask the same gamepad more than once in a frame time (at least
\SI{16}{\ms}).

The request frame is one 0b00000001 byte.
The response is a 4 byte frame representing the simultaneous inputs of the
paddle buttons and joystick, read as follow:

\begin{itemize}
\item Byte 0: see table \ref{byte0}
  \begin{table}
    \begin{tabular}{|*{8}{C|}}
      \hline
      \multicolumn{1}{|l}{7}&\multicolumn{6}{c}{}&\multicolumn{1}{r|}{0}\\
      \hline
      A&B&Z&Start&DPad up&DPad down&DPad left&DPad right\\
      \hline
    \end{tabular}
    \caption{Input byte 0}
    \label{byte0}
  \end{table}
\item Byte 1: see table \ref{byte1}
  \begin{table}
    \begin{tabular}{|*{8}{C|}}
      \hline
      \multicolumn{1}{|l}{7}&\multicolumn{6}{c}{}&\multicolumn{1}{r|}{0}\\
      \hline
      Reserved&Reserved&L&R&C up&C down &C left&C right\\
      \hline
    \end{tabular}
    \caption{Input byte 1}
    \label{byte1}
  \end{table}
\item Byte 2: Joystick X axis range from -128 (left position) to 127 (right
  position)
\item Byte 3: Joystick Y axis range from -128 (down position) to 127 (up position)
\end{itemize}

\begin{figure}
  \includegraphics[width=\textwidth]{input_legend.png}
  \caption{Input bits meaning}
  \label{input_legend}
\end{figure}

\begin{figure}
  \includegraphics[width=\textwidth]{some_input.png}
  \caption{Input bits with some buttons pressed and stick tilted}
  \label{some_input}
\end{figure}

\subsection{Mupen64 movie files}
The file format read by the TASBot64 is that of Mupen64-rr, the version of the
Mupen64 emulator modified for the TAS purpose. The files usually take the .m64
extension.

At the beginning of the file is a simple fixed size header containing a magic
number to verify that the file is indeed of the right format, the format
version, whether the TAS was meant to be run with a game save or a snapshot, the
number of gamepads connected and the number of input frames (all numbers are
little endian). It also contains various metadata such as the name of the game
for which the TAS is made, the author of the TAS or a short description.

The input stream starts after the header, at address 0x400 in the file. It is a
serie of 4 bytes input frames, with the same bits layout as in the gamepad
protocol.

Since it is the console that chooses when it needs input, the input frames are
not tagged with any port number or timestamp : they are simply served, in order,
to any port that requests one.

\subsection{SD card}

\subsection{LCD display}
The display we will use, as the majority of LCD character mode displays, include
a Hitachi HD44780 compatible driver chip. This chip is talked to over a
Read/Write (R/W) pin, a Register Select (RS) pin, an 8 pin bus (DB[7..0]) and a
clock pin (E). Inside the chip are some configuration register and memory,
including an Address Counter (AC), a Display Data RAM (DDRAM) and a Character
Generator RAM (CGRAM).

In our case the R/W pin will stay low as we won't need to read back from the
chip. The CGRAM, whose purpose is to store custom characters, won't be used as
well. The chip also features a 4 bits mode, where DB[7..4] is used on two clock
fronts to provide one byte, which won't be used as well.

When RS is high, each falling edge on E write the byte given on DB[7..0] to
DDRAM, at the address stored in the internal register AC, that represents the
cursor position. AC is then incremented (or decremented, according to
configuration), so that the next character is written to the next posision on
the screen.

When RS is low, instead of character data, the byte on DB[7..0] is interpreted
as a command. These commands allow one to set the configuration, clear the
screen or set AC, among others. The variable length high part of the byte
specifies the command, while the low part is the argument.

Each command or memory write takes a defined time to process before another
action is possible, thus a delay of approximately \SI{40}{\us} is necessary
after each action. Refer to the datasheet for more details.

\section{Block diagram}

\section{Choice of components}
\subsection{MCU}
\subsubsection{CPU}
The gamepad protocol requires being ablo to flip a pin each \si{\us} at
most. With the lowest speed available (\SI{40}{\MHz}), that leaves us 40
instructions to decide what to do, which is enough. Console requests happen at
most each \SI{16}{\ms} and the whole exchange takes at most \SI{200}{\us}. We
then have $\SI{16}{\ms} - \SI{200}{\us} = \SI{15.8}{\ms}$ to do all the rest.

Writing to the screen (takes max \SI{6.4}{\ms} overall to fill the whole
screen) can largely be done in parallel to other tasks as it is mostly waiting
and it doesn't even need to be done every frame. All the remaining time can be
used to buffer data from the SD, and even that can probably offloaded in most
part to a DMA channel. So our CPU budget is plenty.

\subsubsection{Memory}

The smallest flash memory available is \SI{16}{\kibi\byte}. However if we look
at Microchip document AN1045, which is about using a library to access FAT32
filesystems, the Table 10 page 11 shows that their lib needs about
\SI{27}{\kibi\byte} of program memory. Moreover this table assumes a 16 bits CPU
while ours is 32 bits and probably has a more space-consuming instruction
set. While \SI{32}{\kibi\byte} may be sufficient (the PIC32 supports the MIPS16
reduced size instruction set), we choose to require \SI{64}{\kibi\byte} of
program memory to be on the safe side.

As a bonus, the \SI{64}{\kibi\byte} chips have more RAM, that can be used to
buffer more input frames and thus be less sensible to a low quality SD card with
long response times.

\subsubsection{Final choice}

To cater to our limited soldering capabilities, we are limited to a SDIP or
SOIC package.

The cheapest model meeting our requirements sold by our supplier is the
PIC32MX130F064B-I/SP. It is also available in a SOIC package but is more
expensive.

\subsection{Display}
We need to display the name of the game, which may be long and contain Japanese
characters, on one line, hence the longer the line the better. We need another
line to display the eventual error codes and progression of the TAS.

The display must also support a 3.3V electric supply and a backlight to indicate
the power-on status of the TASBot.

\section{Preliminary Bill of Materials}
\begin{tabular}{|l|l|l|l|l|}
  \hline
  Description & Qty & Producer & Producer ref & Supplier ref \\
  \hline
  PIC32 & 1 & Microchip & PIC32MX130F064B-I/SP & 2097776 \\
  \hline
  Display & 1 & Midas & MCCOG22005A6W-BNMLWI & 2218944 \\
  \hline
  SD card slot & 1 & Molex & 503182-1853 & 2334075 \\
  \hline
  Up and down buttons & 2 & & &\\
  \hline
  Passthrough switch & 1 & & &\\
  \hline
  Gamepad cable plugs & 4 & & &\\
  \hline
  UART connector & 1 & & &\\
  \hline
  RJ12 connector for ICD3 & 1 & Molex & 85513-5014 & 1388980 \\
  \hline
\end{tabular}

\end{document}

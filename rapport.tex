\documentclass[a4paper,oneside,12pt]{article}

\usepackage[utf8x]{inputenc}
\usepackage[T1]{fontenc}
\usepackage{eurosym}

\title{TASBot64\\Specifications}
\author{rcavadas, abureau,\\cmutti, lgillot-}

\begin{document}
\maketitle

\section{Functional purposes}
\subsection{Overview}
Lancer des TAS sur la N64 (explication de ce qu'est un TAS)

\subsection{Features}
\begin{itemize}
\item Lire depuis SD le format trouvé sur le net
\item Passthrough ?
\item Écrire dans SD ?
\end{itemize}

\subsection{Performances}
Tourner à 60Hz (ne pas dropper de frames). Le passthrough doit ajouter une latence minimale (zéro frame)

\section{User manual}

\section{Protocols}
Faut qu'on reverse viteuf là

\section{Block diagram}

\section{Preliminary Bill of Materials}

\end{document}

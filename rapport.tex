\documentclass[a4paper,oneside,12pt]{article}

\usepackage[utf8x]{inputenc}
\usepackage[T1]{fontenc}
\usepackage{eurosym}

\title{TASBot64\\Specifications}
\author{rcavadas, abureau,\\cmutti, lgillot-}

\begin{document}
\maketitle

\section{Functional purpose}
The TASBot64 is a device that plugs in a Nintendo 64 to play a pre-recorded Tool
Assisted Speedrun (TAS) input file, faking a gamepad.

A TAS is a speedrun (the completion of a video game in the shortest time
possible) whose inputs, that is to say the sequence of buttons pressing and
movements on the gamepad, are given frame-by-frame on an emulator.

Emulators enable techniques such as frame-by-frame operation, precise digital
and analog inputs, re-recording (saving and reloading the entire state of the
game at any moment) or even memory watching and disassembly.

Thanks to this, TASs showcase feats unattainable to the human player and often
feature blatant abuse of the game mechanisms, exploiting bugs in the games that
was not intended to receive such fast and improbable input.

If the emulator is needed to craft the input file, the sequence of inputs itself
however is perfectly valid and theorically could come from a real gamepad. Hence
the goal of the TASBot64, that is to perform a TAS on a real Nintendo 64 instead
of an emulator, thus validating that the TAS does not depend on any emulator
specificity.

\subsection{Features}
\begin{itemize}
\item Behave as a genuine Nintendo 64 gamepad
\item Read and play the Mupen64-rr input file format from a SD card
\item Allow selecting between 4 different files on the SD
\item Display the name of the current selected file
\item Display errors or unsupported features if they occur
\item Allow passthrough of a gamepad to the console to avoid frequent unplugging
\end{itemize}

\subsection{Performances}
Tourner à 60Hz (ne pas dropper de frames). Le passthrough doit ajouter une latence minimale (zéro frame)

\section{User manual}

\section{Protocols}
Faut qu'on reverse viteuf là

\section{Block diagram}

\section{Preliminary Bill of Materials}

\end{document}
